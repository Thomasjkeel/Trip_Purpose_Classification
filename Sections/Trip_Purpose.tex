\subsection{Overview}
Although a wealth of literature exists regarding the classification of transport modes from GPS traces, investigation into the classification of transport purpose has received far less attention (Yazdizadeh et al., 2019). One reason for this is that users are required to manually provide information about why they have made a trip (as a GPS trace and timestamp is not sufficient alone) (Gong et al., 2014). Notably, mode-classification algorithms often only need few key-identifiers such as speed, acceleration and distance (which are recorded automatically without user-input) to have high accuracy (Dabiri \& Heaslip, 2018). This differs from purpose-classification algorithms where some degree of qualitative information about the individual users is needed. Correspondingly, Yazdizadeh et al. (2019) find that mode-classification models are often shown to be more accurate on average than purpose-classification.\\
Of studies that set out to build purpose-classification models, Gong et al. (2014) characterise three distinct types:
\begin{itemize}
\item Rule-based (using rules to match GPS signal and qualitative identifiers)
\item Probabilistic (using the calculated probability of a given purpose);
\item Machine learning.
\end{itemize}
And a selection of key classification models from the literature from each one these types are detailed in Table 1 along with their inputs and accuracy.
\\
\ [TABLE 1]
\\
As highlighted in Table 1, methods employing the use of Random Forest classifiers (RF) are currently the most popular used (Gong et al., 2018). The trend in the literature has been to train RFs with a high number of inputs and then reduce these using the feature importance as indicator of which inputs are pertinent to the model’s performance. It is likely this trend owes to a lack of understanding around the specific combination of dynamics which govern why people make trips – a major gap in the research of trip purpose classification (Meng et al., 2019).
The inputs used in trip purpose models detailed in Table 1 typically include a combination of user-inputted information and underlying spatial (e.g. distance to respondent’s home/work places; POI; Land Usage), temporal (e.g. time of day; day of week) and socio- demographic (e.g. age; gender; occupation) features. The models are shown to vary in accuracy between 43–96.6\% and have been built on a range of different data sizes (7,039– 131,777 trips) on different years and area. As a result, significant uncertainties have been raised around the cross-comparability of trip purpose studies with any findings being tied to specific locations and times (Jahromi et al., 2016).
There is also disparity in the accuracy of the classification models based on individual purpose classes. As shown in Figure 2.1, the models detailed in Table 1 have broadly struggled in classifying shopping and leisure activities versus activities around education, work and returning home. Arguably, shopping and leisure activities may tend to be less temporally and spatially structured compared to work, education and returning home activities (Lin \& Hsu, 2014).
 FIGURE 2.1 
\\
\subsection{Spatial and temporal representation in trip purpose classification models}
enerally, spatial and temporal features have been identified as the key indicators in trip purpose classification (e.g. Zhu et al., 2014; Yadizadeh et al., 2019) as opposed to socio- demographic features. Despite this, spatial and temporal features have not been applied with any uniform standard throughout the literature (Aslger et al., 2018).
In some cases, only the proximity of the start and end points of the trips to local POIs (Points of Interest) and Personal Locations (i.e. Home and Work) are used to infer about the
purpose of a respondents trips (e.g. Kim et al., 2015 \& Ermugun et al., 2017; Table 1). In other cases, closer attention has been paid to reducing the spatial and temporal complexity of the trips, such as generalising these features through clustering. An example of this spatial generalisation is seen in Montini et al. (2014) who build a high performance trip purpose classification model that makes use of clustering algorithms to group origin and destination points of user trips.
A larger variety of spatial information has been integrated in models than temporal information. The wide range of metrics to account for spatial context such as land use, nearby POIs and Foursquare check-ins have outweighed metrics of temporal importance which are restricted to day of week and time of day. There is also less attention on studying the changes in different types of trip purposes based on daily and weekly trends (Meng et al., 2019).

\subsection{Key issues raised by existing trip purpose research}
One major issue in the literature is that there is little investigation into the longer term effects and seasonality of changes to the trip purpose. Xie et al. (2016) find that weather can fundamentally change how people travel, so including weather in any model that seeks to predict travel is vital. Further, it has even been found that seasonality can severely alter which activities (or purposes) people carry out (Gong et al., 2018). Correspondingly, many of Montreal’s Festivals take place during the months of July–September, which has an effect on the activities people carry out within the city during these months (Grimsrud, M. \& El- Geneidy, 2013).
Also evident in the literature is the fact that the modelling procedure has been approached in a range of different ways. Some studies focus on building individual models for each unique trip purpose and others build all-encompassing, multi-class classification models which account for all unique trip purposes at once. Generally, multi-class has been more effective in the literature (Alsger et al., 2018).
Finally, the majority of the studies ignore the underlying class imbalance of the answers selected by respondents relating to why they have made a particular trip. In most studies, the majority of trips are where the respondent has travelled to work or is returning home, as opposed to a minority trips where the respondent has visited shops or hospitals (Meng et al., 2019). One case where class imbalance is considered is in Xiao et al. (2016) who use under-sampling technique to account for the disproportion of these trip purpose categories.

