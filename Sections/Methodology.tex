\subsection{Study Area}
The study area chosen for this project spans the Greater Montreal region in Eastern Canada. To create this study area, a shapefile containing all of Canada’s 54,000 dissemination areas (DAs) – which are the smallest standard geographic area available on the 2016 Canadian census – was retrieved from Statistics Canada (2016). Using QGIS, a spatial intersect was then calculated between all of the DAs and the GPS traces of respondents to the 2017 MTL Trajet survey to select only areas where data there was an overlap. An illustration of Figure 3.1, the result of this selection is a study area of 7,046 DAs which are used in the analysis of this report.

FIGURE 3.1
\\

Two further shapefiles outlining the geographical boundaries of the city of Montreal, Greater Montreal region were retrieved from Canada’s Open Government Portal (Statistics Canada, 2019). To allow the analysis of this project, all geographically-referenced data were re-projected into the Statistics Canada Lambert (or NAD83). This is a Canadian-centric projection with a 1 metre unit (EPSG, 2019). The re-projection of the data was carried out using Python’s Geopandas library. The total of area of the region chosen for this study is 4,279 km2 and the extent of the City and Greater regions of Montreal are shown in Figure 3.2.

FIGURE 3.2
\\

\subsection{Data description and preparation procedure}
\subsubsection{MTL Trajet travel survey 2016–2017}
Data detailing the results of the 2017 MTL Trajet smartphone travel survey carried out within Montreal, Canada between 18th September 2017 and 18th October 2017, was retrieved from the Montreal Open Database (Ville de Montr\'eal, 2017). This data, which is in a GeoJSON format, has already been pre-processed and cleaned and details 185,285 unique trips from 4,425 unique respondents (Ville de Montr\'eal, 2017). Each unique trip in the dataset contains a unique identification number, a user-defined label for the mode and purpose of the trip; a start and end timestamp, and a spatial reference or geometry. An outline and description of these variables are given in Table 3.1.

TABLE 3.1
\\

The geometry of each trip, specifically, contains a collection of line segments (LineString format) derived from the original GPS trace from the user’s smartphone (see 2.3). For this analysis, the geometry has been re-projected from WGS84 into NAD83 using GeoPandas.
All aspects of the data has been translated from French to English and the unique categories of the mode and purpose of the trips are shown in Table 3.2. Note that although the MTL Trajet app allowed respondents to choose any combination of travel mode categories per
trip, it only allowed one category of travel purpose per trip.

TABLE 3.2
\\

\subsubsection{Land use}
...
\subsubsection{Weather data}
...
\subsection{Modelling procedure}
...
\subsubsection{Rule-based}
...
\subsubsection{Nested multi-logit}
...
\subsubsection{Machine learning}
...

