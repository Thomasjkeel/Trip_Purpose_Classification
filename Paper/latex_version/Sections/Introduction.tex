The purposes by which populations use transport networks on a large scale remains an area with a distinct lack of investigation within the broader mobility studies (Yazdizadeh et al., 2019). In the past, this has primarily been due to an absence of large datasets which combine both the geographically coordinates of people’s movement (i.e. a GPS trace) and the activities for why people make these movements (i.e. for Work, Leisure, etc).
In recent years, improvements to GPS within smartphones has provided researchers a new opportunity to study and record the large scale geospatial movement of people (Zhao et al., 2019). Travel survey apps created for smartphones require much less effort from their participants than traditional travel surveys (i.e. where a separate GPS device is required to record movement) (Li et al., 2016). Therefore, it has become increasingly easy to collect qualitative information about movement within a city – including information about how and why people travel.
The ability of smartphone users’ to create a large amount of geographically-referenced data in these travel survey apps can help researchers generate unique insight into transport behaviour at much finer scales than ever before. This form of participatory data creation is known as Volunteered Geographic Information (hereafter, VGI) (after Goodchild, 2007).
Despite the potential to produce more VGI that can be used to generate insight into urban mobility patterns within a city, there are many cities globally that have no form of formal
10
research initiated within them (Attard et al., 2016). One exception to this, is Montreal, Canada, where a number of mobile travel survey applications have been created to study how and why people move along the city’s transport network. This report makes use of the most recent available dataset from one of these studies: The 2017 MTL Trajet travel survey project (Ville de Montr\'eal, 2019). The MTL Trajet project was carried out between 18th September 2017 and 18th October 2017 and is used in this dissertation to following assess the following research questions:

  Main Research Question:
 \begin{itemize}
  \item Can we effectively classify the purpose of trips using spatial and temporal indicators?
\end{itemize}
  
  Sub-Questions:
\begin{itemize}
  \item Which spatial and temporal indicators are most important for the classification of trip purpose?
  \item Which type of classification model is most effective in the classification of trip purpose?
\end{itemize}
	
The following chapters of the report are organised as follows:
Chapter 2 reviews literature relating to trip purpose classification, the use of VGI in mobility studies and the MTL Trajet survey.
Chapter 3 details the steps carried out in the data pre-processing and collection, the development of space and time metrics from the MTL Trajet data, and the set-up for each trip-purpose classification model.
Chapter 4, presents the results from the analysis procedure and compares the performance of the classification models.
Chapter 5 discusses the extent to which the research objectives (set out in 1.1) have been achieved in the results and highlights uncertainty within them the analysis procedure. Finally, Chapter 6, draws conclusion from the research carried out in this project and suggests areas of further research.
