% !TEX TS-program = pdflatex
% !TEX encoding = UTF-8 Unicode

% This is a simple template for a LaTeX document using the "article" class.
% See "book", "report", "letter" for other types of document.

\documentclass[11pt]{article} % use larger type; default would be 10pt

\usepackage[utf8]{inputenc} % set input encoding (not needed with XeLaTeX)


%%% Examples of Article customizations
% These packages are optional, depending whether you want the features they provide.
% See the LaTeX Companion or other references for full information.

%%% PAGE DIMENSIONS
\usepackage{geometry} % to change the page dimensions
\geometry{a4paper} % or letterpaper (US) or a5paper or....
% \geometry{margin=2in} % for example, change the margins to 2 inches all round
% \geometry{landscape} % set up the page for landscape
%   read geometry.pdf for detailed page layout information

\usepackage{graphicx} % support the \includegraphics command and options

% \usepackage[parfill]{parskip} % Activate to begin paragraphs with an empty line rather than an indent

%%% PACKAGES
\usepackage{booktabs} % for much better looking tables
\usepackage{array} % for better arrays (eg matrices) in maths
\usepackage{paralist} % very flexible & customisable lists (eg. enumerate/itemize, etc.)
\usepackage{verbatim} % adds environment for commenting out blocks of text & for better verbatim
\usepackage{subfig} % make it possible to include more than one captioned figure/table in a single float
% These packages are all incorporated in the memoir class to one degree or another...

%%% HEADERS & FOOTERS
\usepackage{fancyhdr} % This should be set AFTER setting up the page geometry
\pagestyle{fancy} % options: empty , plain , fancy
\renewcommand{\headrulewidth}{0pt} % customise the layout...
\lhead{}\chead{}\rhead{}
\lfoot{}\cfoot{\thepage}\rfoot{}

%%% SECTION TITLE APPEARANCE
\usepackage{sectsty}
\allsectionsfont{\sffamily\mdseries\upshape} % (See the fntguide.pdf for font help)
% (This matches ConTeXt defaults)

%%% ToC (table of contents) APPEARANCE
\usepackage[nottoc,notlof,notlot]{tocbibind} % Put the bibliography in the ToC
\usepackage[titles,subfigure]{tocloft} % Alter the style of the Table of Contents
\renewcommand{\cftsecfont}{\rmfamily\mdseries\upshape}
\renewcommand{\cftsecpagefont}{\rmfamily\mdseries\upshape} % No bold!

%%% END Article customizations

%%% The "real" document content comes below...

\title{Can we predict why people travel within a city? A study evaluating the spatial and temporal characteristics of travel purpose classification within Montreal, Canada}
\author{Thomas J. Keel, Huanfa Chen}
\date{} % to remove date

\begin{document}
\maketitle
\hrulefill
\begin{abstract}
The prediction of why people travel when they move across cities remains an area within the broader mobility studies without extensive investigation. Arguably, this has been hindered by: (1) an absence of large datasets which detail the purposes of individual’s travel across a city; (2) the difficulty in accurately representing space and time within models used to predict why people travel across cities. 
Regarding (1), in recent years, Volunteered Geographic Information (VGI) provided by smartphones travel surveys have provided researchers an opportunity to study the attributes characterising urban mobility patterns within a city at increasingly fine temporal and spatial scales. This study makes uses of one such source of VGI: the 2017 MTL Trajet travel survey app – a project with the aim to study how and why people move within the City of Montreal, Canada. Regarding (2), this project builds upon a small body of research to uncover and categorise spatial and temporal interdependencies of GPS data provided from the MTL Trajet project, before assessing the performance of three machine-learning classification models used to classify this GPS data: Random Forests, Support Vector Machines and Artificial Neural Networks. Specifically, these models are built to classify why people travel based on spatial and temporal characteristics of individual trip. 
\end{abstract}
\hrulefill
\section{Introduction}
The purposes by which populations use transport networks on a large scale remains an area with a distinct lack of investigation within the broader mobility studies (Yazdizadeh et al., 2019). In the past, this has primarily been due to an absence of large datasets which combine both the geographically coordinates of people’s movement (i.e. a GPS trace) and the activities for why people make these movements (i.e. for Work, Leisure, etc).
In recent years, improvements to GPS within smartphones has provided researchers a new opportunity to study and record the large scale geospatial movement of people (Zhao et al., 2019). Travel survey apps created for smartphones require much less effort from their participants than traditional travel surveys (i.e. where a separate GPS device is required to record movement) (Li et al., 2016). Therefore, it has become increasingly easy to collect qualitative information about movement within a city – including information about how and why people travel.
The ability of smartphone users’ to create a large amount of geographically-referenced data in these travel survey apps can help researchers generate unique insight into transport behaviour at much finer scales than ever before. This form of participatory data creation is known as Volunteered Geographic Information (hereafter, VGI) (after Goodchild, 2007).
Despite the potential to produce more VGI that can be used to generate insight into urban mobility patterns within a city, there are many cities globally that have no form of formal
10
research initiated within them (Attard et al., 2016). One exception to this, is Montreal, Canada, where a number of mobile travel survey applications have been created to study how and why people move along the city’s transport network. This report makes use of the most recent available dataset from one of these studies: The 2017 MTL Trajet travel survey project (Ville de Montr\'eal, 2019). The MTL Trajet project was carried out between 18th September 2017 and 18th October 2017 and is used in this dissertation to following assess the following research questions:

  Main Research Question:
 \begin{itemize}
  \item Can we effectively classify the purpose of trips using spatial and temporal indicators?
\end{itemize}
  
  Sub-Questions:
\begin{itemize}
  \item Which spatial and temporal indicators are most important for the classification of trip purpose?
  \item Which type of classification model is most effective in the classification of trip purpose?
\end{itemize}
	
The following chapters of the report are organised as follows:
Chapter 2 reviews literature relating to trip purpose classification, the use of VGI in mobility studies and the MTL Trajet survey.
Chapter 3 details the steps carried out in the data pre-processing and collection, the development of space and time metrics from the MTL Trajet data, and the set-up for each trip-purpose classification model.
Chapter 4, presents the results from the analysis procedure and compares the performance of the classification models.
Chapter 5 discusses the extent to which the research objectives (set out in 1.1) have been achieved in the results and highlights uncertainty within them the analysis procedure. Finally, Chapter 6, draws conclusion from the research carried out in this project and suggests areas of further research.

  
\section{Trip purpose classification}
%\subsection{Study areas}
\subsection{Overview}
Although a wealth of literature exists regarding the classification of transport modes from GPS traces, investigation into the classification of transport purpose has received far less attention (Yazdizadeh et al., 2019). One reason for this is that users are required to manually provide information about why they have made a trip (as a GPS trace and timestamp is not sufficient alone) (Gong et al., 2014). Notably, mode-classification algorithms often only need few key-identifiers such as speed, acceleration and distance (which are recorded automatically without user-input) to have high accuracy (Dabiri \& Heaslip, 2018). This differs from purpose-classification algorithms where some degree of qualitative information about the individual users is needed. Correspondingly, Yazdizadeh et al. (2019) find that mode-classification models are often shown to be more accurate on average than purpose-classification.\\
Of studies that set out to build purpose-classification models, Gong et al. (2014) characterise three distinct types:
\begin{itemize}
\item Rule-based (using rules to match GPS signal and qualitative identifiers)
\item Probabilistic (using the calculated probability of a given purpose);
\item Machine learning.
\end{itemize}
And a selection of key classification models from the literature from each one these types are detailed in Table 1 along with their inputs and accuracy.
\\
\ [TABLE 1]
\\
As highlighted in Table 1, methods employing the use of Random Forest classifiers (RF) are currently the most popular used (Gong et al., 2018). The trend in the literature has been to train RFs with a high number of inputs and then reduce these using the feature importance as indicator of which inputs are pertinent to the model’s performance. It is likely this trend owes to a lack of understanding around the specific combination of dynamics which govern why people make trips – a major gap in the research of trip purpose classification (Meng et al., 2019).
The inputs used in trip purpose models detailed in Table 1 typically include a combination of user-inputted information and underlying spatial (e.g. distance to respondent’s home/work places; POI; Land Usage), temporal (e.g. time of day; day of week) and socio- demographic (e.g. age; gender; occupation) features. The models are shown to vary in accuracy between 43–96.6\% and have been built on a range of different data sizes (7,039– 131,777 trips) on different years and area. As a result, significant uncertainties have been raised around the cross-comparability of trip purpose studies with any findings being tied to specific locations and times (Jahromi et al., 2016).
There is also disparity in the accuracy of the classification models based on individual purpose classes. As shown in Figure 2.1, the models detailed in Table 1 have broadly struggled in classifying shopping and leisure activities versus activities around education, work and returning home. Arguably, shopping and leisure activities may tend to be less temporally and spatially structured compared to work, education and returning home activities (Lin \& Hsu, 2014).
 FIGURE 2.1 
\\
\subsection{Spatial and temporal representation in trip purpose classification models}
enerally, spatial and temporal features have been identified as the key indicators in trip purpose classification (e.g. Zhu et al., 2014; Yadizadeh et al., 2019) as opposed to socio- demographic features. Despite this, spatial and temporal features have not been applied with any uniform standard throughout the literature (Aslger et al., 2018).
In some cases, only the proximity of the start and end points of the trips to local POIs (Points of Interest) and Personal Locations (i.e. Home and Work) are used to infer about the
purpose of a respondents trips (e.g. Kim et al., 2015 \& Ermugun et al., 2017; Table 1). In other cases, closer attention has been paid to reducing the spatial and temporal complexity of the trips, such as generalising these features through clustering. An example of this spatial generalisation is seen in Montini et al. (2014) who build a high performance trip purpose classification model that makes use of clustering algorithms to group origin and destination points of user trips.
A larger variety of spatial information has been integrated in models than temporal information. The wide range of metrics to account for spatial context such as land use, nearby POIs and Foursquare check-ins have outweighed metrics of temporal importance which are restricted to day of week and time of day. There is also less attention on studying the changes in different types of trip purposes based on daily and weekly trends (Meng et al., 2019).

\subsection{Key issues raised by existing trip purpose research}
One major issue in the literature is that there is little investigation into the longer term effects and seasonality of changes to the trip purpose. Xie et al. (2016) find that weather can fundamentally change how people travel, so including weather in any model that seeks to predict travel is vital. Further, it has even been found that seasonality can severely alter which activities (or purposes) people carry out (Gong et al., 2018). Correspondingly, many of Montreal’s Festivals take place during the months of July–September, which has an effect on the activities people carry out within the city during these months (Grimsrud, M. \& El- Geneidy, 2013).
Also evident in the literature is the fact that the modelling procedure has been approached in a range of different ways. Some studies focus on building individual models for each unique trip purpose and others build all-encompassing, multi-class classification models which account for all unique trip purposes at once. Generally, multi-class has been more effective in the literature (Alsger et al., 2018).
Finally, the majority of the studies ignore the underlying class imbalance of the answers selected by respondents relating to why they have made a particular trip. In most studies, the majority of trips are where the respondent has travelled to work or is returning home, as opposed to a minority trips where the respondent has visited shops or hospitals (Meng et al., 2019). One case where class imbalance is considered is in Xiao et al. (2016) who use under-sampling technique to account for the disproportion of these trip purpose categories.


  
\section{Volunteered Geographic Information in mobility research}
Volunteered Geographic information is essentially crowd sourced data which is defined formally as the “widespread engagement of large numbers of private citizens [...] in the creation of geographic information” (Goodchild, 2007, p.212). As VGI is, by definition, volunteered, in some cases it gives us the opportunity to study more personal and subjective forms of information than traditional forms of data collection (such as telephone surveys) (Elwood et al., 2012).
Within mobility studies, VGI has enabled us to study movement at increasingly fine spatial and temporal scales, allowing us to better understand travel in urban mobility patterns than

ever before (Arribas-Bel \& Tranos, 2017; Zahabi et al., 2017). This is because research initiated through smartphones has the potential to reach a larger number of people (as more people have smartphones than GPS devices) (Wu et al., 2016). Notedly, smartphones offer a more cost effective solution than GPS devices, which have been traditionally used in mobility research (Gong et al., 2014; Shi et al., 2018).
Improving our understanding of the context surrounding human mobility in a city can even be used in the estimation of travel demand in the longer term (Meng et al., 2019). This is because the modes of travel people use around a city are often tied to socio-demographic characteristics of underlying populations such as employment status and affluence (Zhang \& Cheng, 2019). Through shifts in these characteristics e.g. with gentrification, this can an effect on the travel patterns and activities carried out within some parts of a city (Bricka et al., 2015). VGI gives us the opportunity to witness these patterns within crowd sourced data.
\\
Finally, Li et al. (2016) distinguish between two types of VGI:
\begin{itemize}
\item Participatory – which is the conscious inclusion of data by private citizens (in the
context of this study this may be in-app responses to trip purpose of the MTL Trajet)
\item Opportunistic – which is unconscious inclusion of data (in the context of this study
this may be a GPS trace).
\end{itemize}
The success of mobility research is often dependent on combination of both types being present within a study.

\subsection{Issues with in VGI in mobility studies}
As many forms of volunteered information require that the users share their data themselves, this creates problems of representativeness in VGI (Li et al., 2016). The general trend in most travel-based surveys is that minority a respondents make up the majority of data, as only some people are willing to share their information (Goodchild \& Li, 2012).
There are also problem with geographical representativeness as, VGI tend to be biased towards cities and in richer nations (Hecht \& Stephens, 2014). As such, Miller \& Goodchild (2014) argue that we must be careful when making generalisations about larger populations (than sample size) from any form of VGI. In terms of travel surveys where trip-purpose has been collected, these often tend to emphasise behavioural patterns of people from certain socio-demographics such as people that want to make their trip intentions known (Kim et al., 2015).
Moreover, there are credibility issues with VGI due to a lack of quality control in its creation (Flanagin \& Metzger, 2008; Goodchild, 2013). VGI collected from travel surveys are particular hindered by this, as we do not ultimately know who each one of the individual participants are and whether they have inputted data correctly (Shi et al., 2018). As such, researchers collecting VGI often have to trust that respondents do not purposefully create mis-information (Attard et al., 2016).

\subsection{MTL Trajet}
Despite the potential the potential of VGI, there are many cities globally that have no form of formal research initiated within them (Attard et al., 2016). One exception to this, is in Montreal, Canada where a number of mobility-based travel surveys have been conducted in the last few years. One of these is the MTL Trajet travel survey project.
The MTL Trajet is a large scale mobile phone travel survey app that has been run yearly between Oct-Nov and has been conducted since 2016 (Ville de Montr\'eal, 2019). The app itself is built from the Itinerum platform which is a framework providing researchers a platform to develop their own travel surveys (Yazdizadeh et al., 2019)
The original aim of the MTL Trajet was to better understand transport behaviours within Montreal (MTL Trajet, 2017), although it is repurposed in this report to investigate trip purpose classification. The MTL Trajet survey originally contained information from its users about their personal locations (i.e. work and residency), although these have been removed from the data before being published in Montreal’s Open Data Portal (Hamouni, 2018).
There are two available sources of MTL Trajet dataset available on Montreal’s Open Data Portal, one that contains raw GPS points and one that contains geo-routed GPS traces from the respondents, the latter of which is used in this report (Ville de Montr\'eal, 2019). Specifically, the geo-routed traces were created after feeding the raw GPS points into the Open-source Routing Machine, which maps the users’ trajectory (see Figure 2.2; Patterson \& Fitzsimmons, 2017b). Because of this, there is a degree of unaccountable inaccuracy in the data used is in this report.

FIGURE 2.2
\\

An Itinerum Platform survey app works by employing a geofencing technique, which updates the sampling frequency of the GPS recordings from the smartphone whilst the user is moving (Patterson et al., 2019). When the user stops for more than 120 seconds, the update rate of GPS recordings drops and the user is prompted to end the trip and answer questions relating to how and why they have made the trip (Figure 2.3; Patterson et al., 2019). The app begins recording movement again when the user leaves the geofence once again.

FIGURE 2.3
\\

The MTL Trajet has seen limited use in the literature and has instead mainly been restricted to use within the City of Montreal’s transport department (MTL Trajet, 2017). One example in the literature is in Yazdizadeh et al. (2019) who use the 2016 edition of MTL Trajet survey data for transport purpose classification models (which is described in Table 2.1).

  
\section{Methodology}
\subsection{Study Area}
The study area chosen for this project spans the Greater Montreal region in Eastern Canada. To create this study area, a shapefile containing all of Canada’s 54,000 dissemination areas (DAs) – which are the smallest standard geographic area available on the 2016 Canadian census – was retrieved from Statistics Canada (2016). Using QGIS, a spatial intersect was then calculated between all of the DAs and the GPS traces of respondents to the 2017 MTL Trajet survey to select only areas where data there was an overlap. An illustration of Figure 3.1, the result of this selection is a study area of 7,046 DAs which are used in the analysis of this report.

FIGURE 3.1
\\

Two further shapefiles outlining the geographical boundaries of the city of Montreal, Greater Montreal region were retrieved from Canada’s Open Government Portal (Statistics Canada, 2019). To allow the analysis of this project, all geographically-referenced data were re-projected into the Statistics Canada Lambert (or NAD83). This is a Canadian-centric projection with a 1 metre unit (EPSG, 2019). The re-projection of the data was carried out using Python’s Geopandas library. The total of area of the region chosen for this study is 4,279 km2 and the extent of the City and Greater regions of Montreal are shown in Figure 3.2.

FIGURE 3.2
\\

\subsection{Data description and preparation procedure}
\subsubsection{MTL Trajet travel survey 2016–2017}
Data detailing the results of the 2017 MTL Trajet smartphone travel survey carried out within Montreal, Canada between 18th September 2017 and 18th October 2017, was retrieved from the Montreal Open Database (Ville de Montr\'eal, 2017). This data, which is in a GeoJSON format, has already been pre-processed and cleaned and details 185,285 unique trips from 4,425 unique respondents (Ville de Montr\'eal, 2017). Each unique trip in the dataset contains a unique identification number, a user-defined label for the mode and purpose of the trip; a start and end timestamp, and a spatial reference or geometry. An outline and description of these variables are given in Table 3.1.

TABLE 3.1
\\

The geometry of each trip, specifically, contains a collection of line segments (LineString format) derived from the original GPS trace from the user’s smartphone (see 2.3). For this analysis, the geometry has been re-projected from WGS84 into NAD83 using GeoPandas.
All aspects of the data has been translated from French to English and the unique categories of the mode and purpose of the trips are shown in Table 3.2. Note that although the MTL Trajet app allowed respondents to choose any combination of travel mode categories per
trip, it only allowed one category of travel purpose per trip.

TABLE 3.2
\\

\subsubsection{Land use}
...
\subsubsection{Weather data}
...
\subsection{Modelling procedure}
...
\subsubsection{Rule-based}
...
\subsubsection{Nested multi-logit}
...
\subsubsection{Machine learning}
...


  
\section{Results}
\subsection{Results from clustering}
...

\subsection{Results  from modelling}
Overall, the accuracy of the three trip-purpose classifiers were found to vary in performance between 48.4–69.3\% after 5-fold cross-validation (Table 4.9). We see that the MLP and RF only make predictions on 56.4\% and 60.8\% of the data, respectively. One reason for this is that these multi-class classifiers so they evaluate all the trip-purpose classes at once, as opposed to the SVC which evaluates the classes one-vs-one (see Table 4.9). Notably, the SVC is found to performs the least accurately.
...

TABLE X
\\ 

When broken down by individual trip purposes in Table 4.10 (as shown visually in Figure 4.24), work and returning home trips were overwhelmingly the most accurately trip- purpose. However, this is likely as this was the predominant class in the dataset perhaps leading to overprediction. MLP was shown to have the highest precision rate, with about half of the trip purposes above 0.5. This indicates that this model was slightly more confident when making prediction for these classes. Additionally, RF was found to have the highest recall for Work and Returning Home trips, indicating this model was the most outright accurate with the classification of these.

...
  
\section{Conclusions}
In conclusion, we present an in-depth analysis into the feasibility of using space and time indicators in trip purpose classification modelling and find that they offer some degree of explanation in seemingly chaotic trip purposes.
Despite this, the modelling approach used in this report only focuses on Montreal and only for one month in September. And to this extent the research is frozen in time and limited by space. We can thus assume, this modelling procedure may have a completely different result for other cities. Moving forward, it is clear that trip purpose classification models will need include more contextual information if we hope to correctly identify key indicators of travel purpose. But, indicators themselves will need to be individualised to specific activities and cities.
...
  
\end{document}
